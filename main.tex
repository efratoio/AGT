%! TEX program = xelatex
\documentclass[tikz,dvipsnames]{vldb}
\usepackage{color}
\usepackage{graphicx}
\usepackage{amssymb}
\usepackage{amsmath}
\usepackage{latexsym}
\usepackage{epstopdf}
\usepackage{multicol}
\usepackage{caption}
\usepackage[vlined,ruled,commentsnumbered]{algorithm2e}
\SetKwRepeat{Do}{do}{while}
% \usepackage{tikz-qtree,tikz-qtree-compat}
\usepackage{tikz}
\usetikzlibrary{shapes.geometric} % required for the ellipse shape
\usetikzlibrary{arrows, backgrounds, calc, hobby, positioning,chains,fit,shapes}
% \usetikzlibrary{positioning,chains,fit,shapes,calc,shadows,arrows}
% \usepackage{geometry}

\usepackage{caption}
\usepackage{latexsym}
\usepackage{amssymb}
\usepackage{amsmath}
\usepackage{subcaption}

\usetikzlibrary{shapes.geometric} % required for the ellipse shape
\usetikzlibrary{backgrounds}
\usetikzlibrary{calc,trees,positioning,arrows,fit,shapes,calc}
\tikzset{vertex style/.style={
    draw=#1,
    thick,
    fill=#1!70,
    text=black,
    ellipse,
    minimum width=.1pt,%0.1cm,
    minimum height=0.5cm,
    font=\small,
    outer sep=1pt,
  },
  text style/.style={
    sloped,
    text=black,
    font=\footnotesize,
    above,
    align=center
  }
}

%\usepackage{tikz}
\tikzset{every tree node/.style={align=center, anchor=north}}
\definecolor{myblue}{RGB}{80,80,160}
\definecolor{mygreen}{RGB}{80,160,80}
%multi-row
\usepackage{multirow}
% \usepackage{subfigure}
\usepackage{subcaption}
%\usepackage{subfig}
\usepackage{url}
\usepackage{array}
\let\chapter\undefined

%\usepackage[showframe=true]{geometry}
\usepackage{changepage}

\usepackage{listings}% http://ctan.org/pkg/listings
\lstset{
  basicstyle=\ttfamily,
  %basicstyle=\fontfamily{lmvtt}\selectfont
  mathescape
}


\newcolumntype{M}[1]{>{\centering\arraybackslash}m{#1}}
%\usepackage{subcaption}

\captionsetup{skip=0pt}
%\DeclareCaptionType{copyrightbox}


\newcommand{\reminderb}[1]{\textcolor{blue}{[[[#1]]]}}
\newcommand{\tbd}[1]{TBD: #1}
\newcommand{\reminder}[1]{\textcolor{red}{[[[#1]]]}}
\newcommand{\lotan}[1]{\reminder{\bf (Lotan)~#1}{\typeout{#1}}}
\newcommand{\efrat}[1]{\reminder{\bf (Efrat)~#1}{\typeout{#1}}}
\newcommand{\naama}[1]{\reminderb{\bf (Naama)~#1}{\typeout{#1}}}

%\newcommand{\daniel}[1]{}
%\newcommand{\amir}[1]{}
%\newcommand{\nave}[1]{}

\newcommand{\bbB}{\ensuremath{\mathbb{B}}}
\newcommand{\bbS}{\ensuremath{\mathbb{S}}}
\newcommand{\bbD}{\ensuremath{\mathbb{D}}}
\newcommand{\bbN}{\ensuremath{\mathbb{N}}}
\newcommand{\bbZ}{\ensuremath{\mathbb{Z}}}
\newcommand{\bbR}{\ensuremath{\mathbb{R}}}
\newcommand{\BX}{\ensuremath{\bbB[X]}}
\newcommand{\NX}{\ensuremath{\bbN[X]}}
\newcommand{\NXD}{\ensuremath{\bbN[D]}}
\newcommand{\ZX}{\ensuremath{\bbZ[X]}}
\newcommand{\ND}{\ensuremath{\bbN[D]}}
%\newcommand{\Dom}{\mbox{$\mathbb{D}$}}
%\newcommand{\Bool}{\mbox{$\mathbb{B}$}}
\newcommand{\Real}{\mbox{$\mathbb{R}$}}
%\newcommand{\Nat}{\mbox{$\mathbb{N}$}}
%\newcommand{\tinyNat}{\mbox{\tiny $\mathbb{N}$}}
%\newcommand{\scrNat}{\mbox{\scriptsize $\mathbb{N}$}}
%\newcommand{\scrBool}{\mbox{\scriptsize $\mathbb{B}$}}
%\newcommand{\Natinf}{\Nat^{\infty}}
\newcommand{\Realinf}{\Real^{\infty}}
\newcommand{\Realpminf}{\Real^{\pm\infty}}

\newcommand{\ssum}{\mbox{$\sum\,$}}
\newcommand{\notimes}{\mbox{$\otimes\:$}}
\newcommand{\sM}{\!_M}
\newcommand{\sL}{\!_L}
\newcommand{\sK}{\!_K}
\newcommand{\sH}{\!_H}
\newcommand{\sKprime}{\!_{K'}}
\newcommand{\sLprime}{\!_{L'}}
\newcommand{\sbbN}{\!_\bbN}
\newcommand{\sbbB}{\!_\bbB}
\newcommand{\sbbZ}{\!_\bbZ}
\newcommand{\sN}{\!_N}
\newcommand{\sW}{\!_W}
\newcommand{\sKM}{\!_{K\!\otimes\!M}}
\newcommand{\sKL}{\!_{K\!\otimes\!L}}
\newcommand{\sSS}{\!_{\bbS}}
\newcommand{\supp}{\mbox{supp}}
\newcommand{\Tuples}[2]{#2^{#1}}
\newcommand{\sAbst}{\!_{ab}}


\newcommand{\systemName}{{\tt SpaIrBP}} %sparql inference by provenance 
\newcommand{\union}{{\tt UNION}}
\newcommand{\sprql}{{\tt SPARQL}}

\newcommand{\trio}{Trio(X)}
\newcommand{\lin}{Lin(X)}
\newcommand{\why}{Why(X)}
\newcommand{\trioExample}{Trio-example}
\newcommand{\posBoolExample}{PosBool-example}

\newcommand{\trans}{{parse-to-query-mapping}}
\newcommand{\parse}{{parse-to-prov-mapping}}

\newcommand{\problemname}{\tt TOP-K}
\newcommand\mycommfont[1]{\footnotesize\ttfamily{#1}}
\newcommand\twoheaduparrow{\mathrel{\rotatebox{90}{$\twoheadrightarrow$}}}
\newcommand{\cev}[1]{\reflectbox{\ensuremath{\vec{\reflectbox{\ensuremath{#1}}}}}}
\newtheorem{theorem}{Theorem}[section]
\newtheorem{lemma}[theorem]{Lemma}
\newtheorem{proposition}[theorem]{Proposition}
\newtheorem{corollary}[theorem]{Corollary}
\newtheorem{claim}[theorem]{Claim}
\newtheorem{conjecture}[theorem]{Conjecture}
\newtheorem{example}[theorem]{Example}
\newtheorem{problem}[theorem]{Problem}
\newtheorem{question}[theorem]{Question}
\newtheorem{note}[theorem]{Note}
\newtheorem{remark}[theorem]{Remark}
\newtheorem{observation}[theorem]{Observation}
\newtheorem{definition}[theorem]{Definition}
\newcommand{\tup}[1]{\mathbf{#1}}
 \makeatletter
  \let\@copyrightspace\relax
  \makeatother



\let\OLDthebibliography\thebibliography
\renewcommand\thebibliography[1]{
  \OLDthebibliography{#1}
  \setlength{\parskip}{0pt}
  \setlength{\itemsep}{0pt plus 0.3ex}
}

%definition of proof sketch
\makeatletter
\DeclareRobustCommand{\qed}{%
  \ifmmode % if math mode, assume display: omit penalty etc.
  \else \leavevmode\unskip\penalty9999 \hbox{}\nobreak\hfill
  \fi
  \quad\hbox{\qedsymbol}}
\newcommand{\openbox}{\leavevmode
  \hbox to.77778em{%
  \hfil\vrule
  \vbox to.675em{\hrule width.6em\vfil\hrule}%
  \vrule\hfil}}
\newcommand{\qedsymbol}{\openbox}
\newenvironment{proofsketch}[1][\proofname]{\par
  \normalfont
  \topsep6\p@\@plus6\p@ \trivlist
  \item[\hskip\labelsep\itshape
    #1.]\ignorespaces
}{%
  \qed\endtrivlist
}
\newcommand{\proofname}{Proof Sketch}
\makeatother
%end definition of proof sketch

\SetAlFnt{\small}

\usepackage[pdfborder={0 0 0}, plainpages, pdfpagelabels=false, pdfstartview=FitH]{hyperref}
\usepackage{paralist}
\begin{document}
\title{Adaptive Seeding in Social Networks}
\author{Efrat Abramovitz, Naama Boer, Lotan Faygler \\ Tel Aviv University}


% \affil{Computer Science Department, Tel Aviv University} \\

\maketitle

%\begin{abstract}


%\end{abstract}




\vspace{-1mm}
% %!TEX root = main.tex
\section{introduction}\label{sec:intro}

%!TEX root = main.tex
\subsection{motivation for social seeding}



Influence maximization refers to the field of study of the various algorithmic
 approaches for selecting a group of early adopters of some product or idea,
 in a way that will result in a large cascade in the social network.
 It was first formulated by Domingos and Richardson in 2001 and later developed




Two major issues with the influence maximization model motivated Singer
 in 20XX [] to come up with a somewhat different approach to the problem.
 The first problem was narrowing down to a limited core sample of users
 in practice.
 For example, in marketing applications merchants can only access users
 who engaged with the brand somehow in the past.
 In other cases, profile-based targeting can bring to a significant reduction
 in the size of the targeted group.
 From these reasons and others, one often ends up applying the methods on
 a small core sample of the users, therefore decreasing the chance of having
 influential users among them.
 Without influential users in the sample, influence maximization techniques
 are likely to result in poor outcomes.
 Since the heavy-tailed degree distribution of social network, in a small
 random sample there is low probability to find high degree nodes, and since
 many of the influence maximization techniques rely on selecting such nodes,
 application of these methods tends to be ineffective when applied to a
 small core set of users.

The second issue with the influence maximization mechanism is its underlying
 assumption that nodes that receive an incentive forward the information
 with probability 1, whereas nodes who do not receive an incentive forward
 the information with probability determined by the according to the diffusion
 model.

In an attempt of better suiting the above mentioned challenges, the adaptive
 seeding model was formed.
 Instead of spending all the budget on a subgroup of nodes from the available
 (and often small) core set, only a part of the budget is spent on them
 in the first stage, and in the second stage the remaining part is spent
 on their influential friends, who hopefully have arrived.
 In other words, adaptive seeding aims to recruit influential neighbors
 to forward become early adopters by allocating a part of the budget for
 that purpose, as opposed to influencing them to forward the information
 without incentives.

\subsection{The friendship paradox}


The fundamental premise of the adaptive seeding model is that the friends
 of the core set are more influential than the set itself, an empirical
 phenomena first discovered in 1991 [Feld 1991].
 This premise, if correct, justifies the adaptive seeding model's two stage
 approach.
 The friendship paradox states that the degree of a node is bounded from
 above by the average degree of its neighbors, with constant probability.
 In Feld's paper from 1991, he provides a simple analysis that gives a strong
 intuition to the validity of the paradox, by showing a bound on the expected
 degree of a node against the expected degree of a random neighbor in the
 graph.

However, this does not suffice to establish the friendship paradox.
 In [], Singer shows how to construct examples in which Feld's intuition
 holds, but the friendship paradox does not.

In 2015, Lattanzi and Singer proved that the friendship paradox holds for
 small samples in power-law perturbed graphs.
 
\begin{definition}[Power law]
Power law graph is a social network graph in which the probability to observe a node
 with degree i is proportional to $i^{-\beta}$, for a constant $\beta$.

\end{definition}

\begin{definition}[perturbed power law graph]

 Given a graph G whose degree distribution is power law and some $p\in [0,1]$, its \textbf{perturbed power law graph} G(p) is the graph G where every one of the edges in G is rewired with probability $p$.
\end{definition}

\begin{theorem}

 For any perturbed power-law graph, with constant probability there is an
 asymptotic gap between the average degree of a random set of polylogarithmic
 size and the average degree of its set of neighbors.
 [Lattanzi and Singer 2015]
\end{theorem}


Since many classes of social network graphs are contained in the perturbed
 power-law family, this last result lays out a firm base to the adaptive
 seeding algorithmic approach.
 By favoring the adaptive seeing two-stage approach over the traditional
 influence maximization approach, the expected number of nodes influenced
 can be asymptotically larger.



[if we want we can add a really nice sketch that demonstrates the friendship
 influence in a perturbed power law graph and a non perturbed power law
 graph - gives great intuition]




%!TEX root = main.tex

\subsubsection{NP-Hardness}
It is well known that finding $k$-element set that maximizes monotone and submodular function in NP hard, with an easy reduction to MAX-COVER, but as mentioned before simple greedy algorithm obtains $(1-\frac{1}{e})$ approximation. \cite{nemhauser1978analysis}. This approximation is also tight \cite{feige1998threshold}.
NP-Hardness proofs for the independent cascading model and linear threshold model were already presented in 
Kempe et al. work \cite{kempe2003maximizing}. 

In general, the adaptive seeding model is more complex then the calssic influence maximization problem, since every influence maximization problem can be simulated by the adaptive seeding model, if $S=V$.




% \input{framework}
%!TEX root = main.tex
\section{introduction}\label{sec:intro}

%!TEX root = main.tex
\subsection{motivation for social seeding}



Influence maximization refers to the field of study of the various algorithmic
 approaches for selecting a group of early adopters of some product or idea,
 in a way that will result in a large cascade in the social network.
 It was first formulated by Domingos and Richardson in 2001 and later developed




Two major issues with the influence maximization model motivated Singer
 in 20XX [] to come up with a somewhat different approach to the problem.
 The first problem was narrowing down to a limited core sample of users
 in practice.
 For example, in marketing applications merchants can only access users
 who engaged with the brand somehow in the past.
 In other cases, profile-based targeting can bring to a significant reduction
 in the size of the targeted group.
 From these reasons and others, one often ends up applying the methods on
 a small core sample of the users, therefore decreasing the chance of having
 influential users among them.
 Without influential users in the sample, influence maximization techniques
 are likely to result in poor outcomes.
 Since the heavy-tailed degree distribution of social network, in a small
 random sample there is low probability to find high degree nodes, and since
 many of the influence maximization techniques rely on selecting such nodes,
 application of these methods tends to be ineffective when applied to a
 small core set of users.

The second issue with the influence maximization mechanism is its underlying
 assumption that nodes that receive an incentive forward the information
 with probability 1, whereas nodes who do not receive an incentive forward
 the information with probability determined by the according to the diffusion
 model.

In an attempt of better suiting the above mentioned challenges, the adaptive
 seeding model was formed.
 Instead of spending all the budget on a subgroup of nodes from the available
 (and often small) core set, only a part of the budget is spent on them
 in the first stage, and in the second stage the remaining part is spent
 on their influential friends, who hopefully have arrived.
 In other words, adaptive seeding aims to recruit influential neighbors
 to forward become early adopters by allocating a part of the budget for
 that purpose, as opposed to influencing them to forward the information
 without incentives.

\subsection{The friendship paradox}


The fundamental premise of the adaptive seeding model is that the friends
 of the core set are more influential than the set itself, an empirical
 phenomena first discovered in 1991 [Feld 1991].
 This premise, if correct, justifies the adaptive seeding model's two stage
 approach.
 The friendship paradox states that the degree of a node is bounded from
 above by the average degree of its neighbors, with constant probability.
 In Feld's paper from 1991, he provides a simple analysis that gives a strong
 intuition to the validity of the paradox, by showing a bound on the expected
 degree of a node against the expected degree of a random neighbor in the
 graph.

However, this does not suffice to establish the friendship paradox.
 In [], Singer shows how to construct examples in which Feld's intuition
 holds, but the friendship paradox does not.

In 2015, Lattanzi and Singer proved that the friendship paradox holds for
 small samples in power-law perturbed graphs.
 
\begin{definition}[Power law]
Power law graph is a social network graph in which the probability to observe a node
 with degree i is proportional to $i^{-\beta}$, for a constant $\beta$.

\end{definition}

\begin{definition}[perturbed power law graph]

 Given a graph G whose degree distribution is power law and some $p\in [0,1]$, its \textbf{perturbed power law graph} G(p) is the graph G where every one of the edges in G is rewired with probability $p$.
\end{definition}

\begin{theorem}

 For any perturbed power-law graph, with constant probability there is an
 asymptotic gap between the average degree of a random set of polylogarithmic
 size and the average degree of its set of neighbors.
 [Lattanzi and Singer 2015]
\end{theorem}


Since many classes of social network graphs are contained in the perturbed
 power-law family, this last result lays out a firm base to the adaptive
 seeding algorithmic approach.
 By favoring the adaptive seeing two-stage approach over the traditional
 influence maximization approach, the expected number of nodes influenced
 can be asymptotically larger.



[if we want we can add a really nice sketch that demonstrates the friendship
 influence in a perturbed power law graph and a non perturbed power law
 graph - gives great intuition]




%!TEX root = main.tex

\subsubsection{NP-Hardness}
It is well known that finding $k$-element set that maximizes monotone and submodular function in NP hard, with an easy reduction to MAX-COVER, but as mentioned before simple greedy algorithm obtains $(1-\frac{1}{e})$ approximation. \cite{nemhauser1978analysis}. This approximation is also tight \cite{feige1998threshold}.
NP-Hardness proofs for the independent cascading model and linear threshold model were already presented in 
Kempe et al. work \cite{kempe2003maximizing}. 

In general, the adaptive seeding model is more complex then the calssic influence maximization problem, since every influence maximization problem can be simulated by the adaptive seeding model, if $S=V$.




%!TEX root = main.tex
\section{preliminaries}\label{sec:prelim}
%!TEX root = main.tex
\subsection{influence maximization}\label{sec:inf}

The first attempt to model the cascading influence of individuals in a social network in an algorithmic problem was posed by Domingos and Richardson. They introduced the idea that the profit of marketing to an individual may be composed not only from the sales done to that individual, but also from the "network value" - the sales to other individuals that are influenced by him \cite{domingos2001mining}. This idea was inspired by viral marketing strategies, and was presented as an algorithmic problem.
In their celebrated seminal work, Kempe, Kleinberg and Tardos \cite{kempe2003maximizing} introduce new models to capture influence, models that were inspired from mathematical sociology and interacting particles systems. This work also provides an hardness proof to the problem of selecting influential nodes and presented a class of influence functions that represents the dynamics of adoption, hence suggesting an operational approach to the problem. This is the class of submodular functions. 
Kempe et al. further provided computational experiments that shows that algorithms designed specifically to solve the problem of influence maximization as modeled by them, out perform classical node selection heuristics. These heuristics, like degree centrality and distance centrality are well established node centrality measures in the field of network graph analysis.

\subsubsection{influence maximization model}
Denote a directed network graph $G(V,E)$ - where nodes represent individuals. A node can be active or inactive. Active nodes can influence their inactive out-neighbors. This influence may be stronger as more neighbors of inactive node becomes active. The influence on an active node's neighbors can be modeled in various ways:

\textbf{Linear Threshold Model}
Each node $v$ is assigned with a threshold value $\theta_v\in[0,1]$. Each edge from some neighbor $w$ to $b$ has a weight $b_{v,w}$, s.t. 
$\sum\limits_{w\in N(v)}{b_{v,w}}\le 1$ . $G(V,E)$ is a network graph, with random thresholds for each node, weights assigned to the edges as described, and some start set $A_0$ of active nodes. The diffusion process progress in discrete steps, where in time $t$ the active set $A_t$ contains all nodes in the set $A_{t-1}$ and also every node $v$ that realizes the condition: $$\sum\limits_{w\in N(v)}{b_{v,w}}>\theta_v$$ The process ends when $A_{t+1}=A_t$


\textbf{Independent Cascade Model}
In this model we assign each edge $(w,v)$ a probability $p_{v,w}$ when a node $w$ becomes active, it is given one chance to activate $v$ with probability $p_{v,w}$. Each attempt to activate one of the neighbors of $w$ will occur once - if $w$ was activated in time $t-1$ then it will try to activate its neighbors in time $t$, and never again. Each attempt will succeed with probability $p_{v,w}$, and the probability is calculated independently to every activation trial event. The process ends when no more trials are allowed.


In the two models we have an initial active set $A_0$ that causes a series of other nodes activations over time. We can now define the influence function $$f:2^V\rightarrow \mathbb{R}_+$$ that gives for each set $A\subseteq V$ its expected number of nodes that will be active at the end of the process. The influence maximization problem is defined for a parameter $k$: which $k$ maximizes $f(A)$ s.t. $|A|=k$.

It is shown \cite{kempe2003maximizing} that this optimization problem is NP hard, and it was shown that for linear threshold and independent cascade models, an optimal solution can be approximated with a factor of $(1-\frac{1}{e}-\varepsilon)$. The algorithm that obtained this result is a greedy hill-climbing algorithm. 


\begin{definition}[Submodular function]
A function $f:2^U\rightarrow \mathbb{R}_+$ where $U$ is some ground set, is submodular if for each $u\in U$ and $S\subseteq T\subseteq U$ it holds that $$f(S\cup\{u\})-f(S)\ge f(T\cup\{u\})-f(T)$$ A monotone submodular function, is a function as described that is also monotone i.e. for every $u\in U$ and $S\subseteq U$ $$f(S\cup\{u\})\ge f(S)$$
\end{definition}

Kempe et al. proved that the independent cascade model and the linear threshold model are indeed submodular and monotone in a sense that the influence function they define is submodular. Finding a $k$-element subset of a ground set that maximizes submodular function is NP-hard, but Nemhauser et al.  have proved that it can be approximated within a factor of $(1-\frac{1}{e})$ with a greedy hill climbing algorithm  \cite{nemhauser1978analysis}. 

\begin{theorem} \label{thm:nemhauser}
\cite{nemhauser1978analysis} For a non-negative, monotone submodular function $f$, let $S$ be a set of size $k$ obtained by selecting elements one at a time, each time choosing an element that provides the largest marginal increase in the function value. Let $S^*$ be the set that maximizes the value of $f$ across all $k$-element sets. Then $f(S)\ge(1-\frac{1}{e})f(S^*)$. 
\end{theorem}

To overcome the fact that the influence functions for the models mentioned are hard to evaluate - simulating several diffusion processes and sampling them, can be used to obtain arbitrarily close approximations, with high probability. Since our main interest in this survey is to explore the adaptive version of the problem, we will show similar usage of the sampling technique in solving the adaptive model that will be introduced later.


The linear threshold model and independent cascade model are sometimes too broad for approximation guarantees, so a new simpler model was introduced by Kempe et al.- the triggering model.
\begin{definition}[Triggering model]
 Each node chooses random and independently a "triggering set" $T_v\subseteq N(v)$ according to some distribution on subsets of it's neighbors. A node will become active at time $t$ if one of the nodes in it's triggering set became active in time $t-1$. A proof that this model is also submodular and monotone was provided \cite{kempe2003maximizing}.
\end{definition}


%!TEX root = main.tex
\subsection{Adaptive Seeding Problem Formulation}\label{sec:model}
The adaptive seeding model is a two-stage stochastic optimization framework. We are given a set of nodes {X} and their neighbors {N(X)}, each associated with a probability $p_i$. In addition we are given a budget $k \in N$ and a function $f:2^{N(X)} \rightarrow \mathbb{R}$.  In the first stage, a set $S \subseteq X$ can be selected, which causes each one of its neighbors to materialize independently with probability $p_i$. In the second stage, the remainder of the budget can be used to optimize the function $f \left( \cdot \right)$ over the realized neighbors. This function quantifies the expected number of individuals in the network that will be influenced as a result of selecting a subset of early adopters. The goal is to select a subset $S \subseteq X$ of size at most $k$ s.t. the function can be optimized in expectation over all possible realizations of its neighbors with the remaining budget $k - |S|$.

A function is \textit{adaptively seeded} if the adaptive seeding model can be approximated within constant factor.

\lotan{insert in mathematics the adaptive approach}


\subsubsection{Knapsack Constraints}
In this variation, the network graph is node-weighted. The goal now is to find a subset of nodes $S \subseteq X$ which can maximize the objective function of the
second stage, in expectation over all realizations of neighbors of $S$ s.t the cost of the second stage is now $k - c(S)$. \lotan{improve}

%!TEX root = main.tex
\subsubsection{Randomized-and-Relaxed Non-Adaptive Policies}\label{sec:nonAdaptive}
In order to get an adaptive solution, most of the algorithms use a fractional, non-adaptive relaxation to the problem. A policy is called non adaptive if already in the first stage, before the nodes in the second stage realize, the policy commits on
both $S \subseteq X$ and $Q \subseteq N(S)$ it will select. The ratio between the expected value of the optimal adaptive policy and the expected value of the optimal non
adaptive policy is unbounded \cite{seeman2013adaptive}, therefore randomized-and-relaxed
non adaptive policies are needed. Such policies only commit to the probability they will select each node that appears in the second stage. Such a policy is a set $S \subseteq X$ and a distribution $q$ over $N(S)$ that describes the probability the algorithm will
select $i \in N(s)$ in the second stage, if it realizes. The optimal randomized-and-relaxed problem can be formulated as following:

\[\max_{S \subseteq X} \sum_{T \subseteq N(X)}{\left(\prod_{i \in T}{p_i q_i}\prod_{i \not\in T}{(1-p_i q_i)}\right)f(T)} \]
\[\textup{s.t.}\quad  |S|+ \sum_{i \in N(S)}{p_i q_i} \le k \]

While working on a problem with different costs, the budget constraint becomes 
\[ c(S) + \sum_{i \in N(S)}{c(i)p_i q_i} \leq B \]
The ratio between an optimal adaptive solution and an optimal randomized-and-relaxed non adaptive one is called the adaptivity gap of the problem.

%!TEX root = main.tex
\section{models of adaptive seeding}\label{sec:models}
There are some varition of the adpative seeding problem that were studied. In this section we will observe some of them, the lower and upper bounds that were found, and talk about algorithms that wereinvented to approximate these models.
%!TEX root = main.tex
\subsection{Approximation Algorithm for the Knapsack Problem}\label{sec:knapsack}
In this section we will discuss the approximation algorithm presented in [second paper] for an adaptive policy that gives a constant factor $ (\approx 0.0259)$ approximation for the optimal solution that can be found in polynomial time. The section is composed from two parts: in the first part, we will present an optimal non-adaptive approximation algorithm (which is also relaxed and randomized) and in the second part we will present an approximation algorithm for the optimal adaptive policy, which uses the non-adaptive algorithm.
\subsubsection{An Optimal Non-Adaptive Approximation Algorithm}
In this algorithm we will use the notion of \textit{densities} and \textit{marginal density}, therefore, we will begin with some definitions:
\\The \textit{marginal value} of some policy $(H, r)$ to an existing policy $(S, q)$ is denoted by $F_{(S,q)}(H, r)$ and defined as $F_{(S,q)}(H, r) = F(S \cup H, q \vee r) − F(S, q)$. Similarly, the marginal cost is $C_{(S,q)}(H, r) = C(S \cup H, q \vee r) − C(S, q)$. The marginal density of a policy is given by: 
\[
    D_{S,q}(H, r) = 
\begin{cases}
    \frac{F_{S,q}(H, r)}{C_{S,q}(H, r)}& \text{if } (S, q) \neq (H, r)\\
    0              & \text{otherwise}
\end{cases}
\]

Finally, $w$ denotes the expected costs vector, defined as the element-wise multiplication of c and p (i.e. the costs and probabilities of the model).

The algorithm considers groups of $\frac{1}{\epsilon}$ nodes from $X$ and greedily adds $x \in X$ to $S$ as long as there is enough budget left while updating $q$ in a way that maximizes the \textbf{density} of $(S,q)$. The update of the probabilities vector $q$ of $N(X)$ finds an approximately densent addition of first-stage node $x \in X$.  For each $x$, it greedily adds the second-stage neighbors of $x$ that maximize the marginal density, until either the budget is exhausted or until the marginal density can no longer be improved.  In the end it chooses the completed solution which has the maximum value (out of the groups of size $\frac{1}{\epsilon}$ it began with). pseudo-code of the algorithm can be found in [???add pics?] divided to the main algorithm \textbf{DensestAscent} and a procedure \textbf{AddDensest}. 
\begin{figure}[h]
\includegraphics[width=8cm]{KSDA.PNG}
\centering
\end{figure}
\begin{figure}[h]
\includegraphics[width=8cm]{KSAD.PNG}
\centering
\end{figure}
It can be proven that \textbf{AddDensest} returns a $(1 - 1/e)$-approximation to the maximal marginal density. This approximation can be translated to a $(1 - 1/e^{1 - 1/e} - \epsilon)$-approximation guarantee for \textbf{DensestAscent} and overcoming the knapsack constrains complication by considering the fact that we tested all subsets $S \subseteq X$ of size $1/\epsilon$. It can be shown that for an optimal solution $(O, v)$ \textbf{DensestAscent} returns $(S, q)$ such that $F(S, q) \ge (1 − 1/e^{1−1/e})(1 − \epsilon)F(O, v)$ and by that proving the following theorem: 
\begin{theorem} For any constant $\epsilon > 0$, \textbf{DensestAscent} is a $(1 - 1/e^{1 - 1/e} - O(\epsilon))$-approximation algorithm of the optimal non-adaptive policy for general submodular functions.
\end{theorem}

\subsubsection{Adaptive Policies}
In this section we will present an adaptive policy that is based on the algorithm from the previous section. There are two main points that need to be considered when using a relaxed non adaptive policy as an adaptive one. The first one is that we need to show that a non-adaptive policy (relaxed and randomized) is a good approximation to an adaptive policy. This we can get as a consequence from a Lemma in [A. Badanidiyuru, C. Papadimitriou, A. Rubinstein, L. Seeman, and Y. Singer. Submodular
adaptive seeding. Manuscript, 2015.]. The lemma proves that the gap is $(1 - 1/e)$, and that it is tight. The second point to be considered is that the objective value we get from the nan-adaptive policy can be approximately obtain over the realizations of the nodes in the second stage. This point is harder to overcome as we need to divide the budget between the first and second stages apriori before the nodes in the second stage are being realized. We will handle it by first solving it with a apecial case, in which the cost of each node is not too high.
\textbf{small costs}
In this part we assume that that no node as a cost which is a constant ratio of the total budget. This is a reasonable assumption, as while different nodes can have much different costs, it is very unlikely that a single node will cost a fraction of the total budget. This assumtion makes it possible to convert non-adaptive policies into adaptive ones, with a small loss. We will prove this by using two lemmas (given without proves, proves can be found in[]):
lemma 5.3, lemma 5.4.
present theorem 5.2 and prove it.
\textbf{arbitrarily high costs}
The main idea for overcoming the obstacle of the two stages nature of the problem with adaptive policy that is based on a non-adaptive policy is to consider the optimal adaptive policy as several restricted policies, where each one of these restrictions can be approximated either by using the non-adaptive relaxation, or through an adaptive policy which can be found using sampling and standard techniques for submodular maximization.



the two points needed to be considered
small costs: two lemmas without proofs, theorem 5.2 and proof
general case: introduction, nice, opttc, opttp, all three 
%!TEX root = main.tex
\subsection{submodular}\label{sec:sub}
In 2013 it was already shown that function in the triggering model, a class of monotone submodular function, can be adaptively seeded \cite{seeman2013adaptive}. However, there are other types of models of submodular functions used to describe influence in networks. In 2015 \cite{badanidiyuru2016locally} an algorithm was suggested to handle the entire class of submodular functions.

\subsubsection{outline}
We will see that the non-adaptive scenario approximates well, by using dense blocks - i.e. sets of neighbors that there contribution for the (marginal) influence is large, in porportion to their cardinality. We then note that, if approximation of this dense block can be obtained in the adaptive scenario - we can get good approximation. 	

\subsubsection{Notations}
\textbf{adaptive model} Given a set of nodes $X\subseteq V$ and their set of neighbors $\mathcal{N}(X)$ each associated with probability $p_i$, a budget $k\in\mathb{N}$ and an influence function $f:2^{\mathcal{N}(X)}\rightarrow \mathbb{R}$, we would like to find a set $S\subseteq X$, that will cause it's neighbors to materialize with probability $p_i$, thus allowing to select yet $k-|S|$-elements set $T$ from the realized nodes that will maximize $f(T)$.
\textbf{submodularity}
We will define the function $$f_T(A):=f(T\cup A)$$, since it captures the notion of marginal value with respect to sime set already selected.
 
\subsubsection{non-stochastic version}
In order to understand the intuition behind the algorithm, we will first describe a non-stochastic model of the influence maximization problem. In this version, again we have a set $X\subset V$ of nodes that we can activate, and in this version the nodes neighbors will be activated with probability one. We want to choose a set $T\subseteq S$ of cardinality $t\le k$, and then $k-t$ elements of $\mathcal{N}(T)$, that will maximize the influence function. 

\textbf{First solution}
Simple algorithm will select greedily the best node out of $\mathcal{N}(X)$ and then, if nessecary will also insert
 the node in $X$ that is connected to that best node, i.e. the node that contribute the most to the marginal. In the worst case, $k/2$ best nodes will be selected, and $k/2$ nodes in $X$. 
At least $k/2$ of the nodes we chose gave the best marginal, and since the optimal solution can't do better then choosing $k$ nodes from $\mathcal{N}(X)$ that maximizes marginal, and also adding nodes to existsing set can only decrease their marginal, because of submodularity - we have $1/2$ approximation to the greedy algorithm, and from \ref{thm:nemhauser} we know that the greedy algorithm approximate submodular monotone functions with a factor of $1-\frac{1}{e}$ - Hence this algorithm has $\frac{1-1/e}{2}$ approximation to the optimal solution.

\textbf{Second solution}
From the friendship paradox we can assume that the more influencial nodes will be in $\mathcal{N}(X)$, so we would like to select more nodes from the neighbors of $X$. Set of neighbors and the node from $X$ connected to them will be called \textit{blocks}. Intuitively we want the "best" blocks. This notion of best will be defined to be \textit{densest}. The densest black is the block that obtains the latgest influence in proportion to it cardinality. The marginal density of a block $(x,B)$ in respect to 
$$D_T(B)=f_T(B)/(1+|B|)$$ where the $+1$ comes from the fact that the parent of the block (the node associated to it in $X$) also must be added to the selected set.

 A variation of the last algorithm is to search, for each node in $X$, the densest $\varepsilon$ block of its neighbor. In each iteration, the densest $\varepsilon$ blocks of neighbors is added to the result set (with the parent node). This algorithm gives a $(1-1/e-\varepsilon)$ approximation.


\subsubsection{adaptivity gap}
In the stocastic version the solution is adaptive. Since adaptive solution are hard to compute, non-adaptive policy will approximate the adaptive solution.
 
The gap between adaptive and non-adaptive policy is $(1-\varepsilon)$, and the gap is tight. Reminder: if the non-daptive policies must select at most $k$ nodes the gap is unbounded, thus we use a relaxed version on non-adaptive, random-and-relaxed, that allows to choose $k$ elements in expectation. We will only add the tightness proof, due to lack of space.
\textfb{tight (1-$\varepsilon$ adaptivity gap)}
Consider the instance where $X$ is a single node, connected to $n=1/\delta^2$ nodes with probability $\delta>0$
The influence function will be $f(T)=0$ if $T=\emptyset$, else $f(T)=1$, the budget is $k=2$.
The adaptive policy will choose the node in $X$, wait for the realization, and with probability $1$ at least one node will realize - hence the adaptive policy will obtain $1$. The best non-adaptive policy is to again seed $X$ and the use the rest of the budget in $1/\delta$ nodes in $\mathcal{N}(X)$. The expected utility will be $$1-(1-\delta)^{1/\delta}\approx 1-1/e$$

\subsubsection{stochastic version}
In the non-stochastic version we described an algorithm that gives a $(1-1/e-\varepsilon)$ approximation. In that algorithm we used an approximation of the densest set of neighbors with $\varepsilon$ block, and got  $(1-1/e-\varepsilon)$ approximation. In the stochastic version we can choose blocks of nodes with some probability and use the same solution, but the main issue is to find the densest $\varepsilon$ blocks.
We first assume that some blackbox FindOptimalNonAdaptiveBlock can $\alpha$-approximate the densest $\varepsilon$ blocks.


\IncMargin{2em}
\begin{algorithm}[h]
	\SetKwInOut{Input}{input}\SetKwInOut{Output}{output}
	\LinesNumbered
	\SetKwFunction{FindBestMatch}{FindBestMatch}
	\Input{$f:2^{\mathcal{N}(X)}\rightarrow \mathbb{R}_+$, budget $k$}
	$S\gets\emptyset$, $q\gets 0$ 
	\While{$|S|+\sum\limits_{j\in T}{p_j\le k-\frac{3}{\varepsilon}}$}
	{
		$(x,B)\gets$ FindOptimalNonAdaptiveBlock$(S,T)$
		$(S,T) = (S\cup x,T\cup B)$
	}

	\Return $(S,T)$
	\caption{NonAdaptiveGreedy}\label{algo:nonAda}
\end{algorithm}\DecMargin{1em}

\begin{lemma} \label{lemma:approxEpsilon}
$\forall \varepsilon >0$, assume that in every iteration FindOptimalNonAdaptiveBlock returns a block which is an $\alpha$ approximation to the optimal $\varepsilon$-block. Then, when $k=\Omega(1/\varepsilon^2) algorithm \ref{algo:nonAda} returns a solution $(S,T)$ s..t. $F(T)\ge (1-1/e^{\alpha}-O(\varepsilon))OPT_{NA}$
\end{lemma}

Lemma \ref{lemma:approxEpsilon} shows that we can adaptively seed submodular function,  if we could approximate densest $\varepsilon$ blocks (in polynomial time). 

%!TEX root = main.tex

\section{private cases}
\subsection{private cases - simpler approaches from paper 1}

Given a network $G=(V,E)$, an influence function $f:2^{V}\rightarrow\mathbb{R}_{+}$we
define the following function $F_{H,t}:2^{X}\rightarrow R_{+}$, for
a subgraph $H$ and parameter $t\in\mathbb{N}$ as: $F_{H,t}(S):=\max_{T\subseteq N(S):|T|\leq k}f(S\cup T)$. 

If $F_{H,t}(\cdot)$ is monotone submodular for all $t\leq k$ and
all graphs $H$, Sample Average Approximation (SAA) is a simple technique
which obtains an approximately $(1-1/e)$-approximation. The adaptive
seeding two stage problem can be formulated as the following optimization
problem: $\max\left\{ \sum_{H\in G}p_{H}F_{H,t}(S):S\subseteq X,|S|+y\leq k\right\} $.$\mathcal{G}$
is the set of all possible realizations and $p_{H}$ is the probability
of the bipartite graph $H=(X\cup R,E_{X,R})$ to realize where $R\subseteq N(X)$,
$E_{X,R}$ being the set of edges between $X$ and $R$ in the network
$G=(V,E)$. $F_{t}(S)=\sum_{H\in G}p_{H}F_{H,t}(S)$ is a monotone
submodular function since it is a weighted sum of monotone submodular
functions for any $t\leq k-1$. $F_{t}(S)$ can be approximated by
sampling polynomially many graphs (as demonstrated in the triggering
model chapter of this survey), and the greedy algorithm for submodular
maximization by Nemhauser can be used to obtain a $(1-1/e-\epsilon)$-approximation
for any $\epsilon>0$. By repeating this procedure for every $t\in\{0,\ldots,k-1\}$
and returning the solution $S\subseteq X$ with the maximal value,
a $(1-1/e-\epsilon)$-approximation for the two stage problem can
be obtained. 

/

In {[}Singer and Seeman 2013{]} two special cases of influence functions
for which $F_{H,t}(\cdot)$ is monotone submodular are discussed.
The first is \textbf{additive functions}. A function is additive when
$f(S)=\sum_{i\in S}w_{i}$ for some fixed $w_{i},i\in[n]$. The well
studied Voter model {[}34{]}, for example, is an additive influence
function {[}19{]}. For such functions $F_{H,t}(\cdot)$ as defined
above is monotone submodular. 

The second case mentioned in {[}Singer and Seeman 2013{]} is \textbf{symmetric
submodular functions}. A function is symmetric submodular when $f(T)=\sum_{i\leq T}r_{i}$
where $r_{i}\in\mathbb{R}_{+}$ is some fixed set that respects $r_{1}\geq r_{2}\geq\ldots\geq r_{n}$,
in which case the function $F_{H,t}(\cdot)$ is submodular. The proofs
that additive functions and symmetric submodular functions are submodular
are specified in the full version of the paper {[}Singer and Seeman
2013{]}.


%!TEX root = main.tex
\subsection{Additive Functions for the Knapsack Problem}\label{sec:additiveKnapsack}
In this section we will show how to solve the problem under knapsack constrains for the case that the influence function is additive, i.e. $f(T) = \sum_{i\in T}{f(i)}$. The approach is to solve directly the adaptive case, and we will get $(1-1/e)$-approximation, which is optimal.

First, let's consider a relaxation of the problem:
\[H_{R,B_2}=\max_{q} \sum_{i=1}^{n}{v_i q_i} \]
\[\textup{s.t.}\quad  \sum_{i=1}^{n}{c_i q_i} \le B \]
\[\quad q \in [0,1]^n \]
Where $R$ are the realized nodes from $N(X)$ and $B_2 \in [0,B]$ is the budget left for the second stage.
This is a monotone submodular function for any realization $R \in N(X)$ and every any budget $B_2 \in [0,B]$. Hence, the function $H_{B_2}(S) = \sum_R{p_R H_{R,B2}(S)}$ is also monotone submodular, since submodular functions are closed under addition. For every set $S$ we can evaluate $H_{B_2}(S)$ to within any desired accuracy by sampling. Thus, it can be solved by the algorithm from [refernece "A note on maximizing a submodular set function subject to a
knapsack constraint"] to a $1-1/e - \epsilon$-approximation solution to every value of $B_2$. 
\begin{lemma}
For any $\epsilon > 0$ and any $0 < B_2 < B$ there is a $(1 - 1/e - \epsilon)$ approximation algorithm for:
\[ max_S \{H_{B_2} : S \subseteq S ; c(S) + B_2 \le B\}\]
\end{lemma}
We now show that given a solution to this problem with $B_2$ we have an adaptive policy with a loss of at most $max\{1/2, 1 - \delta_{B_2}\}$ where $\delta_{B_2}$ is the ratio of the highest cost of any second stage element and $B_2$. Without loss of generality an optimal fractional solution has only one fractional entry. Thus we can get at least half the value by taking either only the element associated with the fractional entry or taking all the other elements in the solution. Moreover, by the linearity of the function we know that the contribution of that element to the value of the solution can’t be more than $(1 - \delta_{B_2})$ of the solution’s value and thus by removing it we get an integral solution of value $(1 - \delta_{B_2})$ of the fractional solution in each realization. So we use the same set $S$ returned by the algorithm but restrict the second stage to using only integral solutions, then in all the realizations there is an integral solution of
value at least $max\{1/2, 1 - \delta_{B_2}\}$ of the optimal fractional solution. Thus, the adaptive policy gets at least $max\{1/2, 1 - \delta_{B_2}\}$ of the value of $H_{B_2}(S)$.
%!TEX root = main.tex
\section{results}\label{sec:results}



\bibliographystyle{abbrv}
{\large
\bibliography{bibShort}
}
%\newpage
%\appendix
%\input{letter}



\end{document}
