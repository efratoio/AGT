%!TEX root = main.tex
\subsection{influence maximization}\label{sec:inf}

The first attempt to model the cascading influence of individuals in a social network in an algorithmic problem was posed by Domingos and Richardson. They introduced the idea that the profit of marketing to an individual may be composed not only from the sales done to that individual, but also from the "network value" - the sales to other individuals that are influenced by him \cite{domingos2001mining}. This idea was inspired by viral marketing strategies, and was presented as an algorithmic problem.
In their celebrated seminal work, Kempe, Kleinberg and Tardos \cite{kempe2003maximizing} introduce new models to capture influence, models that were inspired from mathematical sociology and interacting particles systems. This work also provides an hardness proof to the problem of selecting influential nodes and presented a class of influence functions that represents the dynamics of adoption, hence suggesting an operational approach to the problem. This is the class of submodular functions. 
Kempe et al. further provided computational experiments that shows that algorithms designed specifically to solve the problem of influence maximization as modeled by them, out perform classical node selection heuristics. These heuristics, like degree centrality and distance centrality are well established node centrality measures in the field of network graph analysis.

\subsubsection{influence maximization model}
Denote a directed network graph $G(V,E)$ - where nodes represent individuals. A node can be active or inactive. Active nodes can influence their inactive out-neighbors. This influence may be stronger as more neighbors of inactive node becomes active. The influence on an active node's neighbors can be modeled in various ways:

\textbf{Linear Threshold Model}
Each node $v$ is assigned with a threshold value $\theta_v\in[0,1]$. Each edge from some neighbor $w$ to $b$ has a weight $b_{v,w}$, s.t. 
$\sum\limits_{w\in N(v)}{b_{v,w}}\le 1$ . $G(V,E)$ is a network graph, with random thresholds for each node, weights assigned to the edges as described, and some start set $A_0$ of active nodes. The diffusion process progress in discrete steps, where in time $t$ the active set $A_t$ contains all nodes in the set $A_{t-1}$ and also every node $v$ that realizes the condition: $$\sum\limits_{w\in N(v)}{b_{v,w}}>\theta_v$$ The process ends when $A_{t+1}=A_t$


\textbf{Independent Cascade Model}
In this model we assign each edge $(w,v)$ a probability $p_{v,w}$ when a node $w$ becomes active, it is given one chance to activate $v$ with probability $p_{v,w}$. Each attempt to activate one of the neighbors of $w$ will occur once - if $w$ was activated in time $t-1$ then it will try to activate its neighbors in time $t$, and never again. Each attempt will succeed with probability $p_{v,w}$, and the probability is calculated independently to every activation trial event. The process ends when no more trials are allowed.


In the two models we have an initial active set $A_0$ that causes a series of other nodes activations over time. We can now define the influence function $$f:2^V\rightarrow \mathbb{R}_+$$ that gives for each set $A\subseteq V$ its expected number of nodes that will be active at the end of the process. The influence maximization problem is defined for a parameter $k$: which $k$ maximizes $f(A)$ s.t. $|A|=k$.

It is shown \cite{kempe2003maximizing} that this optimization problem is NP hard, and it was shown that for linear threshold and independent cascade models, an optimal solution can be approximated with a factor of $(1-\frac{1}{e}-\varepsilon)$. The algorithm that obtained this result is a greedy hill-climbing algorithm. 


\begin{definition}[Submodular function]
A function $f:2^U\rightarrow \mathbb{R}_+$ where $U$ is some ground set, is submodular if for each $u\in U$ and $S\subseteq T\subseteq U$ it holds that $$f(S\cup\{u\})-f(S)\ge f(T\cup\{u\})-f(T)$$ A monotone submodular function, is a function as described that is also monotone i.e. for every $u\in U$ and $S\subseteq U$ $$f(S\cup\{u\})\ge f(S)$$
\end{definition}

Kempe et al. proved that the independent cascade model and the linear threshold model are indeed submodular and monotone in a sense that the influence function they define is submodular. Finding a $k$-element subset of a ground set that maximizes submodular function is NP-hard, but Nemhauser et al.  have proved that it can be approximated within a factor of $(1-\frac{1}{e})$ with a greedy hill climbing algorithm  \cite{nemhauser1978analysis}. 

\begin{theorem} \label{thm:nemhauser}
\cite{nemhauser1978analysis} For a non-negative, monotone submodular function $f$, let $S$ be a set of size $k$ obtained by selecting elements one at a time, each time choosing an element that provides the largest marginal increase in the function value. Let $S^*$ be the set that maximizes the value of $f$ across all $k$-element sets. Then $f(S)\ge(1-\frac{1}{e})f(S^*)$. 
\end{theorem}

To overcome the fact that the influence functions for the models mentioned are hard to evaluate - simulating several diffusion processes and sampling them, can be used to obtain arbitrarily close approximations, with high probability. Since our main interest in this survey is to explore the adaptive version of the problem, we will show similar usage of the sampling technique in solving the adaptive model that will be introduced later.


The linear threshold model and independent cascade model are sometimes too broad for approximation guarantees, so a new simpler model was introduced by Kempe et al.- the triggering model.
\begin{definition}[Triggering model]
 Each node chooses random and independently a "triggering set" $T_v\subseteq N(v)$ according to some distribution on subsets of it's neighbors. A node will become active at time $t$ if one of the nodes in it's triggering set became active in time $t-1$. A proof that this model is also submodular and monotone was provided \cite{kempe2003maximizing}.
\end{definition}

