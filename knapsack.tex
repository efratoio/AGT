%!TEX root = main.tex
\subsection{Approximation algorithm for the Knapsack problem}\label{sec:knapsack}
In this section we will discuss the approximation algorithm presented in [second paper] for an adaptive policy that gives a constant factor $ (\approx 0.0259)$ approximation for the optimal solution that can be found in polynomial time. The section is composed from two parts: in the first part, we will present an optimal non-adaptive approximation algorithm (which is also relaxed and randomized) and in the second part we will present an approximation algorithm for the optimal adaptive policy, which uses the non-adaptive algorithm.
\subsubsection{An Optimal Non-Adaptive Approximation Algorithm}
In this algorithm we will use the notion of \textit{densities} and \textit{marginal density}, therefore, we will begin with some definitions:
\\The \textit{marginal value} of some policy $(H, r)$ to an existing policy $(S, q)$ is denoted by $F_{(S,q)}(H, r)$ and defined as $F_{(S,q)}(H, r) = F(S \cup H, q \vee r) − F(S, q)$. Similarly, the marginal cost is $C_{(S,q)}(H, r) = C(S \cup H, q \vee r) − C(S, q)$. The marginal density of a policy is given by: 
\[
    D_{S,q}(H, r) = 
\begin{cases}
    \frac{F_{S,q}(H, r)}{C_{S,q}(H, r)}& \text{if } (S, q) \neq (H, r)\\
    0              & \text{otherwise}
\end{cases}
\]

Finally, $w$ denotes the expected costs vector, defined as the element-wise multiplication of c and p (i.e. the costs and probabilities of the model).

The algorithm considers groups of $\frac{1}{\epsilon}$ nodes from $X$ and greedily adds $x \in X$ to $S$ as long as there is enough budget left while updating $q$ in a way that maximizes the \textbf{density} of $(S,q)$. The update of the probabilities vector $q$ of $N(X)$ finds an approximately densent addition of first-stage node $x \in X$.  For each $x$, it greedily adds the second-stage neighbors of $x$ that maximize the marginal density, until either the budget is exhausted or until the marginal density can no longer be improved.  In the end it chooses the completed solution which has the maximum value (out of the groups of size $\frac{1}{\epsilon}$ it began with). pseudo-code of the algorithm can be found in [???add pics?] divided to the main algorithm \textbf{DensestAscent} and a procedure \textbf{AddDensest}. 
\begin{figure}[h]
\includegraphics[width=8cm]{KSDA.PNG}
\centering
\end{figure}
\begin{figure}[h]
\includegraphics[width=8cm]{KSAD.PNG}
\centering
\end{figure}
It can be proven that \textbf{AddDensest} returns a $(1 - 1/e)$-approximation to the maximal marginal density. This approximation can be translated to a $(1 - 1/e^{1 - 1/e} - \epsilon)$-approximation guarantee for \textbf{DensestAscent} and overcoming the knapsack constrains complication by considering the fact that we tested all subsets $S \subseteq X$ of size $1/\epsilon$. It can be shown that for an optimal solution $(O, v)$ \textbf{DensestAscent} returns $(S, q)$ such that $F(S, q) \ge (1 − 1/e^{1−1/e})(1 − \epsilon)F(O, v)$ and by that proving the following theorem: 
\begin{theorem} For any constant $\epsilon > 0$, \textbf{DensestAscent} is a $(1 - 1/e^{1 - 1/e} - O(\epsilon))$-approximation algorithm of the optimal non-adaptive policy for general submodular functions.
\end{theorem}

\subsubsection{Adaptive Policies}
%!TEX root = main.tex
\subsection{Additive Functions for the Knapsack Problem}\label{sec:additiveKnapsack}
In this section we will show how to solve the problem under knapsack constrains for the case that the influence function is additive, i.e. $f(T) = \sum_{i\in T}{f(i)}$. The approach is to solve directly the adaptive case, and we will get $(1-1/e)$-approximation, which is optimal.

First, let's consider a relaxation of the problem:
\[H_{R,B_2}=\max_{q} \sum_{i=1}^{n}{v_i q_i} \]
\[\textup{s.t.}\quad  \sum_{i=1}^{n}{c_i q_i} \le B \]
\[\quad q \in [0,1]^n \]
Where $R$ are the realized nodes from $N(X)$ and $B_2 \in [0,B]$ is the budget left for the second stage.
This is a monotone submodular function for any realization $R \in N(X)$ and every any budget $B_2 \in [0,B]$. Hence, the function $H_{B_2}(S) = \sum_R{p_R H_{R,B2}(S)}$ is also monotone submodular, since submodular functions are closed under addition. For every set $S$ we can evaluate $H_{B_2}(S)$ to within any desired accuracy by sampling. Thus, it can be solved by the algorithm from [refernece "A note on maximizing a submodular set function subject to a
knapsack constraint"] to a $1-1/e - \epsilon$-approximation solution to every value of $B_2$. 
\begin{lemma}
For any $\epsilon > 0$ and any $0 < B_2 < B$ there is a $(1 - 1/e - \epsilon)$ approximation algorithm for:
\[ max_S \{H_{B_2} : S \subseteq S ; c(S) + B_2 \le B\}\]
\end{lemma}
We now show that given a solution to this problem with $B_2$ we have an adaptive policy with a loss of at most $max\{1/2, 1 - \delta_{B_2}\}$ where $\delta_{B_2}$ is the ratio of the highest cost of any second stage element and $B_2$. Without loss of generality an optimal fractional solution has only one fractional entry. Thus we can get at least half the value by taking either only the element associated with the fractional entry or taking all the other elements in the solution. Moreover, by the linearity of the function we know that the contribution of that element to the value of the solution can’t be more than $(1 - \delta_{B_2})$ of the solution’s value and thus by removing it we get an integral solution of value $(1 - \delta_{B_2})$ of the fractional solution in each realization. So we use the same set $S$ returned by the algorithm but restrict the second stage to using only integral solutions, then in all the realizations there is an integral solution of
value at least $max\{1/2, 1 - \delta_{B_2}\}$ of the optimal fractional solution. Thus, the adaptive policy gets at least $max\{1/2, 1 - \delta_{B_2}\}$ of the value of $H_{B_2}(S)$.