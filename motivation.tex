\subsection{motivation for social seeding}



Influence maximization refers to the field of study of the various algorithmic
 approaches for selecting a group of early adopters of some product or idea,
 in a way that will result in a large cascade in the social network.
 It was first formulated by Domingos and Richardson in 2001 and later developed




Two major issues with the influence maximization model motivated Singer
 in 20XX [] to come up with a somewhat different approach to the problem.
 The first problem was narrowing down to a limited core sample of users
 in practice.
 For example, in marketing applications merchants can only access users
 who engaged with the brand somehow in the past.
 In other cases, profile-based targeting can bring to a significant reduction
 in the size of the targeted group.
 From these reasons and others, one often ends up applying the methods on
 a small core sample of the users, therefore decreasing the chance of having
 influential users among them.
 Without influential users in the sample, influence maximization techniques
 are likely to result in poor outcomes.
 Since the heavy-tailed degree distribution of social network, in a small
 random sample there is low probability to find high degree nodes, and since
 many of the influence maximization techniques rely on selecting such nodes,
 application of these methods tends to be ineffective when applied to a
 small core set of users.

The second issue with the influence maximization mechanism is its underlying
 assumption that nodes that receive an incentive forward the information
 with probability 1, whereas nodes who do not receive an incentive forward
 the information with probability determined by the according to the diffusion
 model.

In an attempt of better suiting the above mentioned challenges, the adaptive
 seeding model was formed.
 Instead of spending all the budget on a subgroup of nodes from the available
 (and often small) core set, only a part of the budget is spent on them
 in the first stage, and in the second stage the remaining part is spent
 on their influential friends, who hopefully have arrived.
 In other words, adaptive seeding aims to recruit influential neighbors
 to forward become early adopters by allocating a part of the budget for
 that purpose, as opposed to influencing them to forward the information
 without incentives.

\subsection{The friendship paradox}


The fundamental premise of the adaptive seeding model is that the friends
 of the core set are more influential than the set itself, an empirical
 phenomena first discovered in 1991 [Feld 1991].
 This premise, if correct, justifies the adaptive seeding model's two stage
 approach.
 The friendship paradox states that the degree of a node is bounded from
 above by the average degree of its neighbors, with constant probability.
 In Feld's paper from 1991, he provides a simple analysis that gives a strong
 intuition to the validity of the paradox, by showing a bound on the expected
 degree of a node against the expected degree of a random neighbor in the
 graph.

However, this does not suffice to establish the friendship paradox.
 In [], Singer shows how to construct examples in which Feld's intuition
 holds, but the friendship paradox does not.

In 2015, Lattanzi and Singer proved that the friendship paradox holds for
 small samples in power-law perturbed graphs.
 
\begin{definition}[Power law]
Power law graph is a social network graph in which the probability to observe a node
 with degree i is proportional to $i^{-\beta}$, for a constant $\beta$.

\end{definition}

\begin{definition}[perturbed power law graph]

 Given a graph G whose degree distribution is power law and some $p\in [0,1]$, its \textbf{perturbed power law graph} G(p) is the graph G where every one of the edges in G is rewired with probability $p$.
\end{definition}

\begin{theoreme}

 For any perturbed power-law graph, with constant probability there is an
 asymptotic gap between the average degree of a random set of polylogarithmic
 size and the average degree of its set of neighbors.
 [Lattanzi and Singer 2015]
\end{theoreme}


Since many classes of social network graphs are contained in the perturbed
 power-law family, this last result lays out a firm base to the adaptive
 seeding algorithmic approach.
 By favoring the adaptive seeing two-stage approach over the traditional
 influence maximization approach, the expected number of nodes influenced
 can be asymptotically larger.



[if we want we can add a really nice sketch that demonstrates the friendship
 influence in a perturbed power law graph and a non perturbed power law
 graph - gives great intuition]



