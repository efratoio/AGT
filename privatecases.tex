%!TEX root = main.tex

\section{private cases}
\subsection{private cases - simpler approaches from paper 1}

Given a network $G=(V,E)$, an influence function $f:2^{V}\rightarrow\mathbb{R}_{+}$we
define the following function $F_{H,t}:2^{X}\rightarrow R_{+}$, for
a subgraph $H$ and parameter $t\in\mathbb{N}$ as: $F_{H,t}(S):=\max_{T\subseteq N(S):|T|\leq k}f(S\cup T)$. 

If $F_{H,t}(\cdot)$ is monotone submodular for all $t\leq k$ and
all graphs $H$, Sample Average Approximation (SAA) is a simple technique
which obtains an approximately $(1-1/e)$-approximation. The adaptive
seeding two stage problem can be formulated as the following optimization
problem: $\max\left\{ \sum_{H\in G}p_{H}F_{H,t}(S):S\subseteq X,|S|+y\leq k\right\} $.$\mathcal{G}$
is the set of all possible realizations and $p_{H}$ is the probability
of the bipartite graph $H=(X\cup R,E_{X,R})$ to realize where $R\subseteq N(X)$,
$E_{X,R}$ being the set of edges between $X$ and $R$ in the network
$G=(V,E)$. $F_{t}(S)=\sum_{H\in G}p_{H}F_{H,t}(S)$ is a monotone
submodular function since it is a weighted sum of monotone submodular
functions for any $t\leq k-1$. $F_{t}(S)$ can be approximated by
sampling polynomially many graphs (as demonstrated in the triggering
model chapter of this survey), and the greedy algorithm for submodular
maximization by Nemhauser can be used to obtain a $(1-1/e-\epsilon)$-approximation
for any $\epsilon>0$. By repeating this procedure for every $t\in\{0,\ldots,k-1\}$
and returning the solution $S\subseteq X$ with the maximal value,
a $(1-1/e-\epsilon)$-approximation for the two stage problem can
be obtained. 

/

In {[}Singer and Seeman 2013{]} two special cases of influence functions
for which $F_{H,t}(\cdot)$ is monotone submodular are discussed.
The first is \textbf{additive functions}. A function is additive when
$f(S)=\sum_{i\in S}w_{i}$ for some fixed $w_{i},i\in[n]$. The well
studied Voter model {[}34{]}, for example, is an additive influence
function {[}19{]}. For such functions $F_{H,t}(\cdot)$ as defined
above is monotone submodular. 

The second case mentioned in {[}Singer and Seeman 2013{]} is \textbf{symmetric
submodular functions}. A function is symmetric submodular when $f(T)=\sum_{i\leq T}r_{i}$
where $r_{i}\in\mathbb{R}_{+}$ is some fixed set that respects $r_{1}\geq r_{2}\geq\ldots\geq r_{n}$,
in which case the function $F_{H,t}(\cdot)$ is submodular. The proofs
that additive functions and symmetric submodular functions are submodular
are specified in the full version of the paper {[}Singer and Seeman
2013{]}.


%!TEX root = main.tex
\subsection{Additive Functions for the Knapsack Problem}\label{sec:additiveKnapsack}
In this section we will show how to solve the problem under knapsack constrains for the case that the influence function is additive, i.e. $f(T) = \sum_{i\in T}{f(i)}$. The approach is to solve directly the adaptive case, and we will get $(1-1/e)$-approximation, which is optimal.

First, let's consider a relaxation of the problem:
\[H_{R,B_2}=\max_{q} \sum_{i=1}^{n}{v_i q_i} \]
\[\textup{s.t.}\quad  \sum_{i=1}^{n}{c_i q_i} \le B \]
\[\quad q \in [0,1]^n \]
Where $R$ are the realized nodes from $N(X)$ and $B_2 \in [0,B]$ is the budget left for the second stage.
This is a monotone submodular function for any realization $R \in N(X)$ and every any budget $B_2 \in [0,B]$. Hence, the function $H_{B_2}(S) = \sum_R{p_R H_{R,B2}(S)}$ is also monotone submodular, since submodular functions are closed under addition. For every set $S$ we can evaluate $H_{B_2}(S)$ to within any desired accuracy by sampling. Thus, it can be solved by the algorithm from [refernece "A note on maximizing a submodular set function subject to a
knapsack constraint"] to a $1-1/e - \epsilon$-approximation solution to every value of $B_2$. 
\begin{lemma}
For any $\epsilon > 0$ and any $0 < B_2 < B$ there is a $(1 - 1/e - \epsilon)$ approximation algorithm for:
\[ max_S \{H_{B_2} : S \subseteq S ; c(S) + B_2 \le B\}\]
\end{lemma}
We now show that given a solution to this problem with $B_2$ we have an adaptive policy with a loss of at most $max\{1/2, 1 - \delta_{B_2}\}$ where $\delta_{B_2}$ is the ratio of the highest cost of any second stage element and $B_2$. Without loss of generality an optimal fractional solution has only one fractional entry. Thus we can get at least half the value by taking either only the element associated with the fractional entry or taking all the other elements in the solution. Moreover, by the linearity of the function we know that the contribution of that element to the value of the solution can’t be more than $(1 - \delta_{B_2})$ of the solution’s value and thus by removing it we get an integral solution of value $(1 - \delta_{B_2})$ of the fractional solution in each realization. So we use the same set $S$ returned by the algorithm but restrict the second stage to using only integral solutions, then in all the realizations there is an integral solution of
value at least $max\{1/2, 1 - \delta_{B_2}\}$ of the optimal fractional solution. Thus, the adaptive policy gets at least $max\{1/2, 1 - \delta_{B_2}\}$ of the value of $H_{B_2}(S)$.