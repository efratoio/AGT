%!TEX root = main.tex
\subsection{Adaptive Seeding Problem Formulation}\label{sec:model}
The adaptive seeding model is a two-stage stochastic optimization framework. We are given a set of nodes {X} and their neighbors {N(X)}, each associated with a probability $p_i$. In addition we are given a budget $k \in N$ and a function $f:2^{N(X)} \rightarrow \mathbb{R}$.  In the first stage, a set $S \subseteq X$ can be selected, which causes each one of its neighbors to materialize independently with probability $p_i$. In the second stage, the remainder of the budget can be used to optimize the function $f \left( \cdot \right)$ over the realized neighbors. This function quantifies the expected number of individuals in the network that will be influenced as a result of selecting a subset of early adopters. The goal is to select a subset $S \subseteq X$ of size at most $k$ s.t. the function can be optimized in expectation over all possible realizations of its neighbors with the remaining budget $k - |S|$.

A function is \textit{adaptively seeded} if the adaptive seeding model can be approximated within constant factor.

\lotan{insert in mathematics the adaptive approach}


\subsubsection{Knapsack Constraints}
In this variation, the network graph is node-weighted. The goal now is to find a subset of nodes $S \subseteq X$ which can maximize the objective function of the
second stage, in expectation over all realizations of neighbors of $S$ s.t the cost of the second stage is now $k - c(S)$. \lotan{improve}

%!TEX root = main.tex
\subsubsection{Randomized-and-Relaxed Non-Adaptive Policies}\label{sec:nonAdaptive}
In order to get an adaptive solution, most of the algorithms use a fractional, non-adaptive relaxation to the problem. A policy is called non adaptive if already in the first stage, before the nodes in the second stage realize, the policy commits on
both $S \subseteq X$ and $Q \subseteq N(S)$ it will select. The ratio between the expected value of the optimal adaptive policy and the expected value of the optimal non
adaptive policy is unbounded \cite{seeman2013adaptive}, therefore randomized-and-relaxed
non adaptive policies are needed. Such policies only commit to the probability they will select each node that appears in the second stage. Such a policy is a set $S \subseteq X$ and a distribution $q$ over $N(S)$ that describes the probability the algorithm will
select $i \in N(s)$ in the second stage, if it realizes. The optimal randomized-and-relaxed problem can be formulated as following:

\[\max_{S \subseteq X} \sum_{T \subseteq N(X)}{\left(\prod_{i \in T}{p_i q_i}\prod_{i \not\in T}{(1-p_i q_i)}\right)f(T)} \]
\[\textup{s.t.}\quad  |S|+ \sum_{i \in N(S)}{p_i q_i} \le k \]

While working on a problem with different costs, the budget constraint becomes 
\[ c(S) + \sum_{i \in N(S)}{c(i)p_i q_i} \leq B \]
The ratio between an optimal adaptive solution and an optimal randomized-and-relaxed non adaptive one is called the adaptivity gap of the problem.
